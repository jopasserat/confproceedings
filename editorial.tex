\section*{Foreword}

Understanding brain connectivity in a network-theoretic context has shown much promise in recent years. This type of analysis identifies brain organisational principles, bringing a new perspective to neuroscience. At the same time, large public databases of connectomic data are now available. However, connectome analysis is still an emerging field and there is a crucial need for robust computational methods to fully unravel its potential. This workshop provides a platform to discuss the development of new analytic techniques; methods for evaluating and validating commonly used approaches; as well as the effects of variations in pre-processing steps.

\bigskip

Topics covered by the papers presented in this workshop include, but are not limited to:

\begin{itemize}
\item Data processing for network construction (e.g. fMRI preprocessing, brain parcellation/cortical segmentation, quantifying anatomical/functional/effective connectivity, spatio/temporal models)
\item Multimodal processing (e.g. Building joint networks, multimodal parcellation, modelling the interaction between modalities, manifold alignment for connectivity networks )
\item Network-based classification and biomarker identification
\item Analysis for disease detection
\item Longitudinal analysis (developing/ageing connectome)
\item Evaluation/model validation (e.g. Constructing synthetic data, quantitative evaluation measures)
\item Visualisation
\end{itemize}

\bigskip

The academic objective of the workshop is to bring together researchers in medical imaging and neuroscience to discuss the challenges and development of new techniques in brain connectivity analysis (connectomics), as well as their benefits for clinical applications. 



\vfill

\textit{The BACON Program Chairs}

\pagebreak